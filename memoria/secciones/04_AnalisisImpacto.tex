\chapter{Análisis de impacto}
En este capítulo se realizará un análisis del impacto potencial de los resultados obtenidos durante la realización del TFG, en los diferentes contextos para los que se aplique:

\begin{enumerate}
\item[•] Personal
\item[•] Empresarial
\item[•] Social
\item[•] Económico
\item[•] Medioambiental
\item[•] Cultural
\end{enumerate}

En dicho análisis se destacarán los beneficios esperados, así como también los posibles efectos adversos.

Se recomienda analizar también el potencial impacto respecto a los Objetivos de Desarrollo Sostenible (ODS), de la Agenda 2030, que sean relevantes para el trabajo realizado (\href{https://www.un.org/sustainabledevelopment/es/objetivos-de-desarrollo-sostenible/}{ver enlace})

Además, se harán notar aquellas decisiones tomadas a lo largo del trabajo que tienen como base la consideración del impacto.

\section{Personal}
El Trabajo de Fin de Grado ha tenido un gran impacto personal. Marca el fin de mi grado en informática. Ha sido
un trabajo constante durante meses en el que he aprendido cosas nuevas y en el que he aplicado lo aprendido
durante estos cuatro años.

\section{Empresarial}
Los recursos generados en este TFG pueden ser utilizados por cualquier empresa que así lo desee.

\section{Social}
Los recursos también pueden ser utilizados por particulares, que ahora podrán disfrutar de herramientas más
actualizadas para sus consultas.

\section{Medioambiental}
La visualización de datos geográficos facilita estudio de sucesos medioambientales en España.
