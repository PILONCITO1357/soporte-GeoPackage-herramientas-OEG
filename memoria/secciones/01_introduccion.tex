\chapter{Introducción}
%%---------------------------------------------------------
La introducción del TFG debe servir para que los profesores que evalúan el Trabajo puedan comprender el contexto
en el que se realiza el mismo, y los objetivos que se plantean.

\section{Objetivos} El objetivo principal del trabajo es introducir soporte al formato GeoPackage en herramientas
de Linked Data Geográfico desarrolladas por el Grupo de Ingeniería Ontológica. En el OEG se ha venido
tradicionalmente trabajando con el Instituto Geográfico Nacional para la exportación de algunos de sus datos
geográficos a formato Linked Data. Un ejemplo se puede encontrar en la web del Instituto Geográfico Nacional.
\cite{ign}

Recientemente, el Open Geospatial Consortium ha publicado el formato GeoPackage, que tiene el objetivo de
convertirse en un estándar para la representación de datos geográficos. El objetivo de este trabajo es el de dar
soporte GeoPackage para las herramientas normalmente utilizadas para este tipo de tareas.

\begin{itemize} 
    \item Dar soporte GeoPackage a la herramienta Map4RDF. 
    \item Dar soporte GeoPackage a la herramienta GeoKettle y su plugin para transformar a RDF. 
    \item Realizar un procesado completo de todos los datos del IGN para generar este tipo de formato. 
\end{itemize}

\section{Estado del Arte}

\subsection{GIS} Los sistemas de información geográfica son herramientas que permiten almacenar y analizar datos
geoespaciales. Los sistemas digitales actuales permiten realizar consultas interactivas, añadir entradas a las
bases de datos y visualizarlos de manera intuitiva. La información geográfica se puede aplicar a todo tipo de
áreas, entre las que se encuentran la ingeniería, transporte, telecomunicación, economía, sociologia... Debido a
la gran importancia tanto en el sector público como el privado\cite{gis-standards}, los estándares abiertos
cobran importancia por estar disponibles al público, no tener que pagar licencias y ser consensuados por
organizaciones de estándares internacionales. Entre ellas se encuentra el Open Geospatial Consortium(OEG) que se creó
en 1994 y agrupa a 521 (en marzo de 2021) miembros de organizaciones públicas y privadas.\cite{ogc-members}
El OGC trabaja junto con las principales organizaciones de estándares de su ámbito (ISO/TC 211, W3C,
IETF...) \cite{ogc-whitepaper}

Existen diversos formatos de fichero GIS, divididos en \textbf{raster} y \textbf{vector}. La diferencia es
equivalente a la que existe entre imágenes con resolución limitada por el número de píxeles (raster) y las 
imágenes vectoriales formadas por puntos, líneas y polígonos; con resoluciones infinitas. Cada tipo de formato
tiene sus ventajas y desventajas y la elección dependerá del caso de uso. Existen varios formatos vectoriales
pero para este trabajo sólo se se considerarán el formato \textit{shapefile} y el \textit{GeoPackage} para cumplir
los objetivos.

\subsubsection{Shapefile} El formato ESRI Shapefile (SHP) es un formato de archivo de datos espaciales vectorial
desarrollado por la compañía ESRI a principios de la década de 1990. A pesar de ser propietario, la
especificación es abierta, y se considera un estandar de facto. Debido a su popularidad, goza de grán
compatibilidad con sig. Gracias al uso de un fichero índice, se obtiene una velocidad de lectura alta, y su
eficiencia de tamaño produce archivos relativamente pequeños.

Sin embargo, tiene varias desventajas, algunas derivadas del uso del estándar dBase \cite{shapefile-specs}
\cite{shapefile-no}: 

\begin{enumerate} 
    \item No tiene definición de sistema de referencia de coordenadas \footnote{Un sistema de
            coordenadas es un sistema de referencia que se utiliza para representar la ubicación de entidades
            geográficas, imágenes y observaciones (como las localizaciones GPS) dentro de un marco geográfico
            común. Los sistemas de coordenadas permiten a los datasets geográficos utilizar ubicaciones comunes
            para la integración de datasets. \cite{coordenadas}} 
        , se puede usar uno pero no es parte estándar de la especificación.     
    \item Se reparte en múltiples ficheros: es incómodo y lleva a errores al compartirlos.
    \item Los nombres de atributos están limitados a 10 caracteres ASCII 
    \item El número máximo de campos de atributo es 255.  
    \item Solo admite float, integer, date y text con un máximo de 254 caracteres.  
    \item No se puede especificar conjunto de caracteres de la BBDD.  
    \item El tamaño está limitado a 4GB.  
    \item No admite valores NULL 
    \item No hay forma de describir las relaciones topológicas en el formato.  
    \item Solamente puede almacenar una geometría por archivo.  
    \item Utiliza una estructura de datos de tabla plana, sin jerarquías, relaciones ni estructura en árbol.  
    \item El soporte 3D es muy limitado.  
\end{enumerate}

\subsubsection{GeoPackage}
GeoPackage es un formato GIS abierto, no propietario, basado en estándares, independiente de plataformas,
portable y compacto basado en SQLite; publicado por el OEG en 2014. \cite{geopackage-spec}
    
El formato geopackage tiene las siguientes ventajas \cite{shapefile-no}:
\begin{itemize}

    \item Dispone de índices espaciales basados en R-trees \cite{rtree} que incrementan la velocidad de búsquedas
        espaciales y su visualización en los SIG de escritorio.
       
    \item Todo el contenido se almacena en un único archivo .gpkg.

\end{itemize}


\subsection{Datos enlazados} Explicar qué son los datos enlazados y en específico RDF

\subsection{Portales de Datos abiertos} Poner ejemplo del IGN.

\subsection{Map4RDF}


\subsection{GeoKettle} Explicar qué es GeoKettle y su plugin.

\section{Herramientas de desarrollo} Docker

La imagen de docker de GeoKettle


%%---------------------------------------------------------
