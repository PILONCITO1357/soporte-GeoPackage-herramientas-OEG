\chapter{Introducción}
%%---------------------------------------------------------
La introducción del TFG debe servir para que los profesores que evalúan el Trabajo puedan comprender el contexto
en el que se realiza el mismo, y los objetivos que se plantean.

\section{Objetivos} El objetivo principal del trabajo es introducir soporte al formato GeoPackage en herramientas
de Linked Data Geográfico desarrolladas por el Grupo de Ingeniería Ontológica. En el OEG se ha venido
tradicionalmente trabajando con el Instituto Geográfico Nacional para la exportación de algunos de sus datos
geográficos a formato Linked Data. Un ejemplo se puede encontrar en la web del Instituto Geográfico Nacional.
\cite{ign}

Recientemente, el Open Geospatial Consortium ha publicado el formato GeoPackage, que tiene el objetivo de
convertirse en un estándar para la representación de datos geográficos. El objetivo de este trabajo es el de dar
soporte GeoPackage para las herramientas normalmente utilizadas para este tipo de tareas.

\begin{itemize} \item Dar soporte GeoPackage a la herramienta Map4RDF. \item Dar soporte GeoPackage a la
herramienta GeoKettle y su plugin para transformar a RDF. \item Realizar un procesado completo de todos los
datos del IGN para generar este tipo de formato. \end{itemize}

\section{Estado del Arte}

\subsection{GIS} Los sistemas de información geográfica son herramientas que permiten almacenar y analizar datos
geoespaciales. Los sistemas digitales actuales permiten realizar consultas interactivas, añadir entradas a las
bases de datos y visualizarlos de manera intuitiva. La información geográfica se puede aplicar a todo tipo de
áreas, entre las que se encuentran la ingeniería, transporte, telecomunicación, economía, sociologia... Debido a
la gran importancia tanto en el sector público como el privado\cite{gis-standards}, los estándares abiertos
cobran importancia por estar disponibles al público, no tener que pagar licencias y ser consensuados por
organizaciones de estándares internacionales. Entre ellas se encuentra el Open Geospatial Consortium que se creó
en 1994 y agrupa a 521 (en marzo de 2021) miembros de organizaciones públicas y privadas.\cite{ogc-members}
El OGC trabaja junto con las principales organizaciones de estándares de su ámbito (ISO/TC 211, W3C,
IETF...) \cite{ogc-whitepaper}

Existen diversos formatos de fichero GIS, divididos en \textbf{raster} y \textbf{vector}. La diferencia es
equivalente a la que existe entre imágenes con resolución limitada por el número de píxeles (raster) y las 
imágenes vectoriales formadas por puntos, líneas y polígonos, con resoluciones infinitas. Cada tipo de formato
tiene sus ventajas y desventajas y la elección dependerá del caso de uso. \#TODO ventajas vs desventajas?

\subsubsection{Shapefile} El formato ESRI Shapefile (SHP) es un formato de archivo de datos espaciales
desarrollado por la compañía ESRI. A pesar de ser propietario, la especificación es abierta, y se considera un
estandar de facto.

\subsubsection{GeoPackage} Explicar las ventajas que tiene frente al shapefile para aclarar
por qué merece la pena añadir el soporte a las herramientas.

\subsection{Datos enlazados} Explicar qué son los datos enlazados y en específico RDF

\subsection{Portales de Datos abiertos} Poner ejemplo del IGN.

\subsection{Map4RDF}

\subsection{GeoKettle} Explicar qué es GeoPackage y su plugin.

\section{Herramientas de desarrollo} Docker

La imagen de docker de GeoKettle


%%---------------------------------------------------------
